\documentclass[usenames,dvipsnames,10pt]{beamer}
\usepackage{xcolor}
\usetheme{Warsaw}
\usecolortheme{dove}
\usefonttheme{structuresmallcapsserif}
\useoutertheme{infolines}
\setbeamercolor{alerted text}{fg=ForestGreen}
\setbeamerfont{alerted text}{series=}

\usepackage[utf8]{inputenc}
\usepackage[german]{babel}
\usepackage[T1]{fontenc}
\usepackage{amsmath}
\usepackage{amsfonts}
\usepackage{amssymb}
\usepackage{graphicx}
\usepackage{multicol}
\usepackage{setspace}
%\setbeamercovered{transparent} 
%\setbeamertemplate{navigation symbols}{} 
\usepackage{minted}

\usepackage{listings}
\lstset{escapeinside={<@}{@>}}
\lstdefinestyle{Bash}
{language=bash,
keywordstyle=\color{blue},
basicstyle=\ttfamily,
morekeywords={peter@kbpet},
commentstyle=\color{brown},
morekeywords=[2]{peter@kbpet:},
keywordstyle=[2]{\color{red}},
alsoletter={:~$},
literate={\$}{{\textcolor{BrickRed}{\$}}}1 
         {:}{{\textcolor{BrickRed}{:}}}1
         {~}{{\textcolor{BrickRed}{\textasciitilde}}}1,
} 
%$

\setbeamerfont{itemize/enumerate subbody}{size=\normalsize} %to set the body size
\setbeamerfont{itemize/enumerate subsubbody}{size=\normalsize} %to set the body size
\setbeamertemplate{itemize subitem}{\normalsize\raise1.25pt\hbox{\donotcoloroutermaths$\blacktriangleright$}}  %to set the symbol size


\newcommand{\mysection}[1]{{\bf #1\\\medskip}}
\newcommand{\mysubsection}[1]{\medskip #1\\}
\newcommand{\mybox}{\scalebox{0.5}{\fbox{}}}
\newcommand{\boxes}{
\scalebox{0.5}{\fbox{1}\fbox{2}\fbox{3}…\fbox{37}} \\
}
%\newcommand{\boxes}{{\TINY
%\begin{tabular}{|c|c|c|c|c|}
%\hline 
%\rule[-1ex]{0pt}{2.5ex}   &   &   &   &  \\ 
%\hline 
%\rule[-1ex]{0pt}{2.5ex}   &   &   &   &   \\ 
%\hline 
%\end{tabular} 
%}}

\usepackage{caption}

\author[Michael F. Schönitzer]{Michael F. Schönitzer \\ \href{mailto:michael@schoenitzer.de}{michael@schoenitzer.de}}
\title{Capabilities}
\subtitle{Das vergessene Linux-Security-Feature}

%\logo{} 
%\institute{} 
\date[22. April 2017]{Augsburger Linux-Infotag \\ 22. April 2017}
\begin{document}

\begin{frame}[plain]
\titlepage
\end{frame}


\begin{frame}[<+->]{Der Root User}
	Traditionelles Rechtesystem von UNIX:
	\begin{itemize}
	\item User root mit $UID = 0$ darf alles.
	\item Alle anderen User dürfen \glqq nichts\grqq.
	\end{itemize}
	
	\vspace{1cm}
	
	\begin{block}{Beispiel: NTP-Daemon}
	\begin{itemize}
	\item Muss Zeit setzen \& in Echtzeit arbeiten.
		\begin{itemize}
		\item[$\Rightarrow$] Muss mit Root-Rechten laufen
		\begin{itemize}
		\item[$\Rightarrow$] Darf auch Systemdateien ändern, Festplatten formatieren, Kernelmodule laden, Firewalls verstellen, Prozesse killen…
		\end{itemize}
		\end{itemize}
	\end{itemize}
	\end{block}
\end{frame}

\begin{frame}{Capabilities}
	\begin{block}{Idee}
	Trenne die Rechte feingranularer auf Capabilities auf
	\end{block}
	\pause
	\vfill
	\begin{block}{Beispiel: NTP-Daemon}
	\begin{itemize}
		\item CAP\_SYS\_NICE \\
		Erlaubt die Prozess Priorität zu erhöhen, etc.
		\item CAP\_SYS\_TIME \\
		Erlaubt das Verstellen der Uhrzeit		
	\end{itemize}
	\end{block}
\end{frame}

\begin{frame}[<+->]{Geschichte}
	\begin{itemize}
	\item 1997: POSIX 1003.1e – withdrawn
	\item Linux 2.1
	\item Linux 2.2.11: Capability bounding set
	\item Linux 2.6.24: File Capabilities
	\item Linux 2.6.25: Capability bounding set auf Prozessebene
	\item Linux 2.6.26: Securebits
	\item Linux 2.6.33: Filecapabilities nicht mehr Optional
	\item Linux 3.2%: /proc/sys/kernel/cap\_last\_cap
	\item Linux 3.8
	\item Linux 4.3: Ambient Capabilities
	\end{itemize}
\end{frame}
% Notes:
% Support in ls:
%% https://git.savannah.gnu.org/gitweb/?p=coreutils.git;a=commit;h=84f6abfe00b4ab533145623638b417a2221f9c75


\begin{frame}[t]{Theorie}
	\setstretch{0.4}
	\footnotesize
	
	\begin{multicols}{2}
	
	\mysection{Prozess}
	
	\onslide<2->{\alert<2>{
	\mysubsection{\makebox[12ex][l]{Permitted} $P$}
	\boxes
	}}
	
	\onslide<1->{
	\mysubsection{\makebox[12ex][l]{Effective} \only<1>{$E$}\onslide<2->{\alert<2>{$E\subseteq P$}}}
	\boxes
	}
	
	\onslide<5->{\alert<5>{
	\mysubsection{\makebox[12ex][l]{Inheritable} $I$}
	\boxes
	}}
	
	\onslide<8->{\alert<8>{
	\mysubsection{\makebox[12ex][l]{Ambient} $A$}
	\boxes
	}}
	
	\columnbreak
	\onslide<3->{\alert<3>{
	\mysection{Datei (XAttr)}
	
	\mysubsection{\makebox[12ex][l]{Permitted} $P_{Datei}$}
	\boxes
	
	\mysubsection{\makebox[12ex][l]{Effective} $E_{Datei}$}
	\mybox
	}}
	
	\onslide<5->{\alert<5>{
	\mysubsection{\makebox[12ex][l]{Inheritable} $I_{Datei}$}
	\boxes
	}}
	
	\vfill\onslide<13->{\onslide<1>{*}} % superdirty hack
	\end{multicols}
	
	{
	\hfill
	\begin{minipage}{0.85\textwidth}
	\onslide<4->{\alert<4>{
	\mysection{Kindsprozess}
	
	\mysubsection{\makebox[12ex][l]{Permitted}
	{$P_{Kind} = \only<-11>{P_{Datei}}\only<12->{\alert<12>{\left( P_{Datei} \cap Bset \right)}} \only<7->{\alert<7>{\cup \left( I \cap I_{Datei} \right)}} \only<10->{\alert<10>{\cup A_{Kind}}}$}
	}
	\boxes
	
	\mysubsection{\makebox[12ex][l]{Effective}
	   {$E_{Kind} = E_{Datei} \ ? \ P_{Kind} \only<10->{\alert<10>{\ :\  A_{Kind}}}$}}
	\boxes
	}}
	
	\onslide<6->{\alert<6>{
	\mysubsection{\makebox[12ex][l]{Inheritable} $I_{Kind} = I$}
	\boxes
	}}
	
	\onslide<9->{\alert<9>{
	\mysubsection{\makebox[12ex][l]{Ambient} $ A_{Kind} = \mathrm{(Datei\ priviligiert)} \ ? \ 0 \ : \ A$}
	\boxes
	}}
	\end{minipage}
	\hfill
	}
\end{frame}


\begin{frame}{Securebits flags \& UID changes}
	
	\begin{itemize}[<+->]
	\item \makebox[17ex]{Exec mit SUID bit} $\qquad \Rightarrow \qquad P_{Inheritable} = P_{Effective} = P_{Permitted} = 1$
	\item \makebox[17ex]{Exec mit UID $=0$\hfill} $\qquad \Rightarrow \qquad P_{Inheritable} = P_{Effective} = P_{Permitted} = 1$
	\item \makebox[17ex]{UID $=0 \;\rightarrow\;\; >0$} $\qquad \Rightarrow \qquad P_{Effective} = P_{Permitted} = 0$ 
	\item \makebox[17ex]{UID $>0 \;\rightarrow\;\; =0$} $\qquad \Rightarrow \qquad P_{Effective} = P_{Permitted}$ 
	\end{itemize}
	
	\begin{onlyenv}<5>
	Deaktivierbar durch:
	
		\begin{itemize}
		\item SECBIT\_KEEP\_CAPS
		\item SECBIT\_NO\_SETUID\_FIXUP
		\item SECBIT\_NOROOT
		\item SECBIT\_NO\_CAP\_AMBIENT\_RAISE
		\end{itemize}
	\end{onlyenv}
	\begin{onlyenv}<6>
	Arretierbar durch:
	
		\begin{itemize}
		\item SECBIT\_KEEP\_CAPS{\color{red}\_LOCK}
		\item SECBIT\_NO\_SETUID\_FIXUP{\color{red}\_LOCK}
		\item SECBIT\_NOROOT{\color{red}\_LOCK}
		\item SECBIT\_NO\_CAP\_AMBIENT\_RAISE{\color{red}\_LOCK}
		\end{itemize}
	\end{onlyenv}

\end{frame}


\begin{frame}[fragile]{capability-awareness}
	\begin{itemize}[<+->]
	\item capability-dumb binaries
	\item capability-aware binaries
	\end{itemize}
	
	\usemintedstyle{tango}
	
	\begin{onlyenv}<1>
	Beispiel: 
	
	\begin{lstlisting}[style=Bash]
# old system
$ ls -l /usr/bin/ping
<@\textcolor{black}{-rwsr-xr-x 1 root root 60K 16. Nov 08:57 /usr/bin/ping}@>

# modern system
$ ls -l /usr/bin/ping
<@\textcolor{black}{-rwxr-xr-x 1 root root 60K 16. Nov 08:57 /usr/bin/ping}@>
$ getcap /usr/bin/ping
/usr/bin/ping = cap_net_raw+ep
	\end{lstlisting}
	\end{onlyenv}
	%$
	
	\begin{onlyenv}<2>
	Beispiel:
	
	\begin{listing}[H]
	\captionsetup{labelformat=empty}
	
	\caption{NTPsec: \texttt{ntpd/ntp\_sandbox.c}}
	\begin{minted}[fontsize=\footnotesize]{c}
	#  ifdef HAVE_LINUX_CAPABILITY
	   // ...
	   cap_t caps;
	   char *captext;
	
	   captext = want_dynamic_interface_tracking
	      ? "cap_sys_nice,cap_sys_time,cap_net_bind_service=pe"
	      : "cap_sys_nice,cap_sys_time=pe";
	   caps = cap_from_text(captext);
	   // ...
	   cap_free(caps);
	#  endif  /* HAVE_LINUX_CAPABILITY */
	\end{minted}
	\end{listing}
	
	\end{onlyenv}

\end{frame}
% END Cap Awareness

\begin{frame}{Probleme}
	\begin{block}{CAP\_SYS\_ADMIN}
	\begin{quote}
	It can plausibly be called \glqq the new root\grqq, since on the
	one hand, it confers a wide range of powers, and on the other hand, its broad
	scope means that this is the capability that is required by  many  privileged programs.
		\end{quote}
	\rightline{{\rm --- man CAPABILITIES(7)}}
	\end{block}
\end{frame}

\begin{frame}[fragile]{Probleme}
	\begin{listing}[H]
	\captionsetup{labelformat=empty}
	
	\begin{minted}[fontsize=\footnotesize]{c}
	uid_t euid=geteuid();
	if(euid != 0)
	{
	  // Tell user to run app as root, then exit.
	}

	\end{minted}
	\end{listing}
\end{frame}


\begin{frame}[<+->]{Probleme}
\begin{itemize}
	\item Root User ist Besitzer aller Systemdateien: /etc/shadow, /usr/bin/su, …
	\begin{itemize}
	\item[$\Rightarrow$] Root ohne Capabilities ist exploitable!
	\begin{itemize}
			\item[$\Rightarrow$] Verwende User mit $UID > 0$!
	\end{itemize}
	\end{itemize}
\end{itemize}
\end{frame}


\begin{frame}[fragile]{Probleme}
	Etwa 20 der Capabilities sind Exploitable und erlauben es volle Rechte zu erlangen!\pause
	
	Einfaches Beispiel:
	
	CAP\_FOWNER
	
	\begin{minted}{bash}
	$ chown hacker /etc/shadow
	$ chmod 777 /etc/shadow
	$ vim /etc/shadow
	$ su
	\end{minted}
	\pause
	Betrifft vor allem capability-dump binaries.
	\pause
	Einschränkbar durch Capability-aware binaries, strengere Dateirechte, capability bounding set, Grsec, Selinux, etc…
\end{frame}

\begin{frame}[fragile]{Praktisch}
	\begin{itemize}[<+->]
	\item getcap: Capabilities einer Datei anzeigen
	\item setcap: Capabilities einer Datei setzen
	\item filecap: Capabilities aller Dateien anzeigen
	\item getpcaps:  Capabilities eines Prozesses anzeigen
	\item pscap: Capabilities aller Prozesse anzeigen
	\item capsh: Wrapper, um ein Shell mit Capabilities zu setzen
	\item captest: Testet die Capabilities
	\item libcap: C-libary für Capability-aware libaries
	
	\item ls – Dateien mit Capabilities sind farbig! \pause – Außer auf Debian! \pause
	\item systemd – Capabilities können in Unit-Files spezifiziert werden
	\item pam\_cap – inheritable capabilities beim Anmelden setzen
	\item Docker – nutzt intensiv die Capabilities, anpassbar
	\end{itemize}
	
	\vfill
	\begin{visibleenv}<3>
	\begin{lstlisting}[style=Bash]
	$ find / -user root -perm -4000
	
	$ filecap -a
	\end{lstlisting}
	\end{visibleenv}
\end{frame}

\begin{frame}
\begin{center}
	Vielen Dank für Ihre Aufmerksamkeit!
\end{center}
\end{frame}


\end{document}